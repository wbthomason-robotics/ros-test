\documentclass[11pt]{article}

\usepackage[margin=1.0in]{geometry}
\usepackage[usenames,dvipsnames]{color}
\usepackage{lmodern}
\usepackage[T1]{fontenc}
\usepackage{amsmath}
\usepackage{array}
\usepackage{bigstrut}
\usepackage{booktabs}
\usepackage{enumerate}
\usepackage{fancyvrb}
\usepackage{framed}
\usepackage{graphicx}
\usepackage{hyperref}
\usepackage{ifthen,version}
\usepackage{longtable}
\usepackage{textcomp}
\usepackage{todonotes}
\usepackage{wrapfig}
\usepackage{subcaption}
\usepackage{mwe}
\hypersetup{
    colorlinks,
    %pdfborderstyle={/S/U/W 1}
    linkcolor={red!50!black},
    citecolor={blue!50!black},
    urlcolor={blue!80!black},
}
\usepackage{caption}
\usepackage{subcaption}

\setlength\bigstrutjot{3pt}

\newcolumntype{L}[1]{>{\raggedright\let\newline\\\arraybackslash\hspace{0pt}}m{#1}}

\setlength{\parindent}{0pt}
\setlength{\parskip}{5pt}
\newcommand{\la}{\textlangle{}}
\newcommand{\ra}{\textrangle{}}
\newcommand{\wtodo}[1]{\todo[author=WT,inline]{#1}}

\begin{document}
\title{RPAL ROS \& Programming Skills Assessment}
\date{}
\author{}

\maketitle

\section{Objective:} 

Demonstrate understanding of basic ROS usage and Python (or C++) programming, as well as some basic
robotics concepts. This project will be used to assess your readiness to begin working in RPAL as an
undergraduate researcher.

\section{Collaboration Policy:} 

This assignment \textbf{must be completed alone}. You are permitted to use any internet resources
(short of asking for help on a forum, etc.), books, tools, what have you, but you may \textbf{not}
collaborate with anyone and you may \textbf{not} ask any grad students or anyone else for
assistance. This is an assessment of you and your current skills; if you violate this policy we
cannot make an accurate assessment and will not be able to work with you.

Note that you can come to us if you are truly stuck. If the issue is a bug or simple
misunderstanding with the provided code, we can help you out and you can get back to the assessment.
\textbf{However}, if you come to us and the provided code is working correctly (i.e.\ the error is
with your code), that will constitute termination of the project and we will assess your work in
whatever state it is in.

\section{Running the simulator framework}

We have provided a simulation framework for you to develop and test your code. In order to use the
framework, you'll need to carefully follow these instructions.

\subsection{Setup}

Do these steps \emph{once}, before starting your work. We also recommend using Git to store your
partial progress. Note that successfully completing the setup process is part of the assessment.

\subsubsection{ROS}

Follow the instructions here: \url{https://wiki.ros.org/melodic/Installation} to install ROS
Melodic.

\subsubsection{V-REP}

The simulation framework is based around
\href{http://www.coppeliarobotics.com/downloads.html}{V-REP}, which you can download at that link.
Get the ``Pro EDU'' version for Linux and extract the files to some directory
(\texttt{\textasciitilde/V-REP} is a good choice). Finally, copy the file
\texttt{libv\_repExtRosInterface.so} from this directory to wherever you extracted V-REP.

\subsubsection{Catkin Workspace}

Before starting the project, you need to set up a catkin workspace. To do this, follow the
instructions here: \url{https://wiki.ros.org/catkin/Tutorials/create_a_workspace}. Afterwards, copy
the \texttt{src} subdirectory from this package into \texttt{<YOUR CATKIN ROOT>/src/rpal-ros-test}
and run \texttt{catkin build} to build the initial workspace.

\subsection{Using V-REP}

The second part of this assignment will require you to run V-REP. To this end, we have provided a
V-REP scene file, \texttt{assessment.ttt}, in this directory. Use the following steps to run the
simulator:
\begin{enumerate}
  \item Start \texttt{roscore}.
  \item (assuming you extracted V-REP to \texttt{\textasciitilde/V-REP}), run
  \texttt{\textasciitilde/V-REP/vrep.sh assessment.ttt} in this directory.
  \item Press the "Play" button in the upper toolbar.
  \item Run the commands you need for the particular problem.
  \item Run the commands to execute your code.
\end{enumerate}  

\textbf{To start the simulation, you must hit the play button.}

\section{Assignment:}

RPAL primarily uses ROS and Python for research coding. Thus, you will need to write Python code
using \href{http://wiki.ros.org/rospy}{rospy} to be an effective researcher in the lab. If you need
a refresher for Python, we suggest \href{https://learnxinyminutes.com/docs/python/}{this quick
reference}\footnote{Note that we use Python 3 in most research.}. You should also use the
\href{http://www.numpy.org/}{numpy library} for
\href{https://docs.scipy.org/doc/numpy-1.14.0/reference/generated/numpy.matrix.html}{matrix}
calculations and linear algebra, which is common in robotics research.

If you need a reference for ROS, we suggest the \href{http://wiki.ros.org/ROS/StartGuide}{ROS
``Getting Started'' guide}, the \href{http://wiki.ros.org/rospy}{rospy documentation}, and Jason
O'Kane's \href{https://cse.sc.edu/~jokane/agitr/agitr-letter.pdf}{``A Gentle Introduction to
ROS''}\footnote{Note that AGITR uses C++ instead of Python. The concepts will be the same, but the
language used is different.}.

Implement code to solve the following problems, using the provided simulator framework. Please also
document any difficulties you encountered and how you overcame them in a file named
\texttt{problems.txt}, which you should include with the final submission. When you're done, bundle
up the \emph{whole} \texttt{rpal-ros-test} directory into a \texttt{.tar.gz} and send it to us for
assessment.

\end{document}

%%% Local Variables:
%%% mode: latex
%%% TeX-master: t
%%% End:
